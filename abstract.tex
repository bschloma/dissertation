\abstract{Vast communities of microorganisms inhabit the gastrointestinal tracts of humans and other animals, where they influence diverse aspects of animal health and disease. Our understanding of the types of microbes present in the intestine and the genes that they carry has grown tremendously in recent years, but despite this progress, we are still unable to predict the abundances of microbial strains in the gut and their impact on host phenotypes. This deficiency limits our abilities to uncover causal mechanisms mediating host-microbe interactions and to rationally design novel therapeutic strategies. A major barrier to achieving these goals is our limited ability to experimentally probe the spatial organization of gut bacterial communities, which is thought to be a key driver of microbiota dynamics, but which is largely inaccessible in most systems. This dissertation work addresses these knowledge gaps by combining quantitative theory with controlled experiments in a model system that can uniquely surmount these technical challenges. The larval zebrafish is an optically transparent, model vertebrate that is amenable to live imaging studies, in which bacteria in the gut can be directly visualized and studied \textit{in situ}. Through this approach, we discovered that the biophysical properties of bacteria in the gut, especially their aggregation and swimming behaviors, coupled to intestinal fluid flows, determine in robust but probabilistic ways several large-scale features of whole bacterial populations. These features include global spatial distributions of bacteria throughout the gut, bacterial population dynamics, both at baseline and in response to perturbations like antibiotics, and the ability of bacteria to stimulate immune responses. Through the study and validation of phenomenological models, we argue that these effects are generic and manifest in other animals, including humans, and suggest new strategies to harness these effects for precision microbiome engineering.}