\chapter{Conclusion}

This dissertation has investigated various topics concerning the spatial organization and dynamics of gut bacterial communities using live imaging approaches in larval zebrafish, a model vertebrate. Discussions of specific findings and their impacts are included at the ends of each core chapter. To conclude this dissertation, I end with a brief summary of the ``rules'' inferred in this work, detail how these rules lead to quantitative relationships between common measurements of bacterial populations in the zebrafish gut, and give suggestions for future research.

\section{Biophysical rules of gut bacterial communities}
This work set out to discover general rules governing the spatial organization and dynamics of gut bacterial communities. From examination of 7 different bacterial species and application of various perturbations, we have deduced the following picture for the minimal system of a single bacteria species in an otherwise germ-free larval zebrafish intestine: most bacteria within the gut reside within the intestinal lumen, suspended in fluid. As bacteria divide, one or both daughter cells can either break free, resulting in new planktonic cells, or remain close to the parent cell, resulting in the formation of multicellular, 3D aggregates. These aggregates are likely encased in intestinal mucus, though a complete characterization of the structural elements involved in the formation of aggregates is lacking. As the aggregate continues to grow through cell division, single cells may continue to break out of the aggregate, a process we call fragmentation. Additionally, due to the mixing properties of intestinal fluid flow, distinct aggregates may coalesce together to form a larger one, a process we call aggregation. All the while, intestinal contractions push bacterial aggregates in both directions along  the gut axis, with net transport towards the posterior. Occasionally, aggregates will be expelled from the gut altogether. The expulsion of the largest aggregates registers as abrupt, massive drops in the bacterial abundance within the gut. 

Within this framework, we can now concretely state some rules that have emerged from the work presented here:

\begin{enumerate}
\item 
\textbf{Bacterial aggregation correlates with localization along the length of the gut}. As discussed in Chapter 2 \cite{schlomann_bacterial_2018}, there is a strong correlation between metrics of aggregation behaviors, such as the fraction of the population contained in single cells, and localization along the anterior-posterior axis. The direction of causality has not yet been rigorously established, but perhaps the simplest explanation is that larger aggregates are transported more strongly to the posterior by intestinal contractions. This type of size-dependent transport has been seen in previous studies of particle transport in peristaltic flows \cite{Jimenez-Lozano2009}, and is supported by the motility loss-of-function experiments of Chapter 5 \cite{Wiles2019}, in which motility inhibition results first in aggregation and then a posterior shift in center of mass.

\item 
\textbf{Increased bacterial aggregation leads to larger fluctuations in intestinal abundance due to expulsion}. For strongly aggregated populations, expulsion of the largest aggregate is more severe than it is for a less aggregated population, because there are fewer single cells and small clusters left behind to repopulate the gut. In the face of stochastic expulsion of bacterial aggregates, maintaining a large population of single cells and small clusters can therefore be thought of as a form of bet hedging: a species that puts all of its cells into a single aggregate is dangerously poised for extinction. The large fluctuations in abundance that result from the stochastic expulsion of large aggregates leave a signature on the distribution of abundances measured at a single time point, marked by large variance/mean ratios and left skew towards low abundance outliers.

\item 
\textbf{Antibiotics can lead to larger reductions in aggregated bacteria than planktonic bacteria through intestinal expulsion}. This finding goes against conventional wisdom of antibiotic tolerance, which states that bacterial aggregation is often associated with enhanced tolerance. This association is due to a variety of mechanisms, including physical protection from antibiotic compounds and reduced metabolic activity of cells in biofilms. While these mechanisms are all likely present in the gut, we identified a new mechanism that has the opposite effect: enhanced aggregation and/or reduced growth due to antibiotic exposure can deplete the reservoir population of single bacterial cells in the gut that are required for repopulating after intestinal expulsion events.

\item 
\textbf{Bacterial swimming and chemotaxis can be used to counter aggregation and intestinal expulsion and confer enhanced population stability}. While fast growth and fragmentation alone are sufficient to allow bacterial populations in the gut to maintain stability in the face of expulsion, motility can be an effective means of avoiding aggregation. When combined with chemotaxis toward an anterior-localized cue, motility can also enable bacteria to effectively ``swim upstream'' and remain localized to the anterior gut.

\end{enumerate}
Of note, these rules have a markedly biophysical character. Processes like cell aggregation and intestinal fluid flow determine in no small part a wide variety of key properties of bacterial populations in the gut. The importance of biophysical process to gut bacterial communities revealed here is under-appreciated in the field of microbiota research as a whole, though during the course of this dissertation it has been recognized in a small number of studies from other groups \cite{cremer_effect_2016,moor_high-avidity_2017,tropini2018transient}. It is worth emphasizing that live imaging was absolutely crucial to the discovery of all of these phenomena.

\section{Quantitative relationships between common measurements of zebrafish gut bacterial populations}

Through the use of mathematical models, the above rules can be seen to manifest systematic relationships between various measurements commonly done on resident bacterial populations in the zebrafish gut. Broadly, the relevant measurements can be grouped into three categories: spatial organization, dynamics, and abundance. Measurements of spatial organization involve imaging multiple fish at single time points. Dynamics refers to measuring a single fish's bacterial population over time. Abundance measurements are typically done by dissection and plating of gut contents for many fish at a single time point, and involve no imaging. Combining insights from the entirety of this dissertation, we now have a thorough understanding of how these various metrics are connected, and in several cases, how values of one metric can be quantitatively predicted from measurements of others.

First, we demonstrated that two aspects of spatial organization, the degree of aggregation (sometimes called cohesion) and localization along the length of the gut are highly correlated \cite{schlomann_bacterial_2018}. Specifically, the relationship between the typical location of an aggregate, $x$, and the number of cells it contains, $n$, is roughly logarithmic, $x \sim \log(n)$. This strong correlation has the potential to be extremely useful, because localization, typically quantified by a 1D center of mass, is far easier to measure than the degree of aggregation, typically measured by planktonic fraction. The center of mass can be reliably measured from images obtained with crude, low-magnification fluorescence microscopy. In contrast, measuring planktonic fraction requires accurate identification of bacteria in the gut with single cell resolution, requiring high resolution imaging and sophisticated computational image analysis. The strong correlation identified in \cite{schlomann_bacterial_2018} implies that one can infer the degree of aggregation by measuring center of mass. This connection could be exploited in studies of perturbations that are thought to alter aggregation, which if combined with high-throughput platforms, such as 96-well plates or fluidic systems \cite{logan2018automated}, and automated image analysis, could be done on a large scale.

Second, we rigorously connected the stochastic dynamics of individual populations, measured through time-lapse imaging, with the single-time-point abundance distributions readily measured through dissection and plating. The pseudorandom expulsion of large aggregates from the gut---so-called collapse events---combined with the conventional logistic growth observed between collapses, leads to a mathematical description of population dynamics that has the form of a piecewise deterministic Markov process \cite{hansonBook}: deterministic logistic growth coupled to discontinuous jumps that arrive as a Poisson process. Jumps are multiplicative and characterized by a fraction remaining after collapse, $f$, and the average arrival rate, $\lambda$. This model was quantitatively validated in previous work \cite{wiles_host_2016}, which I was a co-author on. In \cite{wiles_host_2016}, stochastic simulations of this model with parameters measured through time-lapse imaging, including the population growth rate, $r$, and collapse parameters $(f, \lambda)$, were shown to generate abundance distributions consistent with experimental values obtained through dissection and plating. 

In this dissertation work, I furthered this connection with analytic calculations of the model's stationary moments, which completely describe the abundance distribution \cite{schlomann2018stationary}. These analytic results rigorously demonstrated that in the limit of rare but large collapses, relevant for our experimental system, the abundance distribution can be described by a single dimensionless shape parameter, $r^{-1}\lambda\ln f$, which controls the variance and skewness of the distribution (the overall scale of the distribution is set by the carrying capacity, the maximum possible abundance).  This effective shape parameter can be interpreted as the ratio two timescales: the timescale of recovery after a collapse of size $f$, $(r^{-1}\ln f)$, and the timescale of collapse arrival ($\lambda^{-1}$). The upshot of this is that measurements of the abundance distribution can be used to infer the ratio of these two core timescales. Perturbations that decrease the mean abundance but increase the variance and/or skewness can be interpreted as either decreasing the bacterial growth rate relative to the collapse rate, increasing the collapse rate or size, or a combination of these effects. In ongoing, unpublished work by the Guillemin lab, this type of analysis has already been used to track the effects of perturbations to bacterial quorum sensing, which alters bacterial aggregation and therefore collapse size. 

Finally, by developing and validating a kinetic model of how individual bacterial clusters change size, we quantitatively connected measurements of aggregation to collapse size, and therefore to population dynamics and abundance distributions. By introducing rates for bacterial cluster aggregation and fragmentation, this model, introduced in \cite{schlomann_sublethal_2019}, makes precise the notion that populations that are more aggregated will experience larger collapses. The collapses are larger because after the largest cluster gets expelled, there are fewer single cells and small clusters left over. We demonstrated that parameters of this kinetic model can be fit either from abundance distributions, obtained by dissection and plating, or from cluster size distributions, obtained from single time-point images. An analytic expression for how rates of aggregation, fragmentation, and growth determine a typical collapse size are lacking; deriving such an expression would be a useful future endeavor. However, by fitting or directly measuring these rates, collapse sizes can then be measured in simulated trajectories of the population abundance. 

Combining these connections, it is now possible to gain insight into a variety of detailed processes of gut bacterial populations through a few basic measurements. From a few images of fish on a crude fluorescence dissecting microscope and a handful of abundance measurements from dissection and plating, it is now possible to estimate a variety of rate parameters describing in vivo bacterial dynamics that themselves are far more challenging to measure directly. I encourage all future researchers using this system to keep these connections between measurements in mind, as they may well be useful in expediting various inquiries.

\section{Suggestions for future research}

I end by suggesting three avenues for future research:

\begin{enumerate}
\item 
\textbf{Extend the quantitative frameworks for spatial organization and dynamics to multispecies communities}. The results presented here on single species bacterial populations lay the groundwork for quantitative understanding of the spatial organization and dynamics of communities of multiple bacterial species. One specific direction is to study how the spatial mixing of two or more species within aggregates is generated by the kinetics of aggregation, fragmentation, growth, and expulsion. This could be done by imaging multispecies communities with each species labeled by a differently colored fluorophore, computing metrics of spatial correlations, and comparing these metrics to appropriate kinetic models that can be parameterized from measurements. One approach to modeling could be to simply extend the mean-field kinetic model of Chapter 4 \cite{schlomann_sublethal_2019} to a multispecies case and compare statistics of multispecies cluster sizes to image-derived data. Alternatively, one could develop an explicitly spatial model and study metrics like the cross-type pair correlation function, which is already in use in the study of microbial biogeography \cite{welch_spatial_2017}. A quantitative understanding of such spatial patterns could be useful for interpreting imaging data from other host species, such as mice, for which static imaging of multispecies communities in thin slices of fixed gut tissue is becoming increasingly feasible \cite{welch_spatial_2017}. 

\item 
\textbf{Explore the response of gut microbiota to diverse chemical perturbations}. The finding presented here of gut bacterial populations responding strongly to sublethal levels of antibiotics motivates the study of other, non-lethal, chemical perturbations that may alter key biophysical processes such as bacterial aggregation. In particular, antidepressants such as selective serotonin reuptake inhibitors (SSRIs) are known to alter gut microbiota, though underlying mechanisms remain unclear \cite{lukic2019antidepressants}. Studying their effects in our larval zebrafish system through live imaging may reveal unexpected insights. Another relevant class of compounds is antifungals, which are widely prescribed, but whose effects on gut microbiota are poorly understood. Antifungal drugs are often derived from antibiotics but typically lack widespread lethality on bacteria due to their fungi-specific mechanisms of action \cite{ghannoum1999antifungal}. However, given the close relatedness to antibiotics, it is highly plausible that antifungals induce similar non-lethal effects to antibiotics, such as the ones discussed here of modulating growth rates and aggregation behaviors, making them suspect for having disproportionately large in vivo consequences. Of note, the connection between aggregation and anterior-posterior localization discussed above motivates a high-throughput screening approach where many compounds can be rapidly tested for their effects of the center of mass of bacterial populations in the gut, measured for example by low-magnification microscopy in 96-well plates. Specialized plates designed with 45-degree mirrors that enable a side view of larval zebrafish on an inverted microscope would be particularly applicable.

\item 
\textbf{Develop a quantitative understanding of the evolution and transmission of antibiotic resistance in host-microbe metacommunities}. A natural and extremely relevant extension of the antibiotic work presented here is the study of antibiotic resistance. An important and largely open problem in the field is to understand which environments place the strongest selective pressure on drug resistance and facilitate expansion of resistant populations. Is it out in the natural environment, in soil or water sources? Is it in hospitals? At manufacturing sites? Or is it within animal bodies? Understanding which environments are the most important for driving resistance will help optimize policies for combating it. Similarly, how antibiotic resistant organisms spread from host to host---the epidemiology of antibiotic resistance---is a challenging yet extremely relevant problem at present. Larval zebrafish provide a potentially powerful model in which to study these problems, in part due to insights gained during this dissertation work. 

To easily track resistance, I suggest taking a synthetic biology approach inspired by the motility gain-of-function ($\Delta$mot$^{\text{GOF}}$) switches introduced in Chapter 5 \cite{Wiles2019}. Recall that $\Delta$mot$^{\text{GOF}}$ bacteria are normally non-motile, but chemical induction of the genetic switch lifts repression of motility genes, leading to swimming bacteria and expression of a green fluorescent protein (GFP) marker to track switched cells. A fascinating observation made in this study that was not followed up on is that in control experiments of uninduced $\Delta$mot$^{\text{GOF}}$ populations, motile bacteria still eventually emerged in vivo and populated the gut. Subsequent characterization showed that these bacteria mutated parts of the synthetic genetic switch required for motility repression, leading to swimming bacteria that constitutively expressed the GFP marker. These mutations then rapidly swept through the in vivo population, indicating strong selective pressure on swimming motility in the larval zebrafish gut. For comparison, the fraction of mutated cells in the aqueous environment outside the fish never exceeded $\sim$10\%.

I propose that this accidental feature of the gain-of-function switch can be intentionally used to visually map selective pressures in heterogeneous environments. This approach is conceptually the inverse of experimental evolution: there, an organism is placed under a chosen selective pressure and one tracks the traits that respond to this pressure. In the proposed approach, a specific trait is chosen and its evolution is tracked in different environments. By using fluorescent markers of switch induction, as in $\Delta$mot$^{\text{GOF}}$ cells, the selective pressure in each environment can be visually read out.

To study the evolution of antibiotic resistance, one could engineer a similar gain-of-function switch on a resistance gene, effectively ``baiting'' a resistance-conferring mutation . Then, one could colonize a flask of zebrafish, treat with antibiotics, and then track the frequency of resistant, GFP+ clones both inside and outside of the fish. In a similar way, one could study the transmission of antibiotic-resistant bacteria between fish by replacing a fraction of colonized fish with germ-free fish, or by introducing antibiotic-treated fish to an untreated population. With all of the measurements of bacterial populations available in the larval zebrafish system, it should be possible to develop a quantitative theory of evolutionary dynamics and transmission that can make experimentally testable predictions.
\end{enumerate}